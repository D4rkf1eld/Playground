                               % PRÄAMBEL              
\documentclass[10pt]{article}  % article: Typ des LaTeX Dokuments
\usepackage[ngerman]{babel}    % ngerman: Neue deutsche Rechtschreibung (Silbentrennung u.w.) | babel: Ändert die Spracheinstellungen
\usepackage[utf8]{inputenc}    % inputenc: Stellt die Zeicheneingabe fuer Umlaute (in den Editor) sicher
\usepackage[T1]{fontenc}       % T1: Zeichenmenge für westeuropäische Länder | fontenc: Angepasste Zeichendarstellung in der PDF
\usepackage{listings}
                               % PRÄAMBEL

% In der Präambel werden Formate für das gesamte Dokument festgelegt, Pakete geladen und eigene Befehle definiert.
% Zusätzliche Pakete werden mit dem Befehl \usepackage{package} geladen. Damit wird meist eine weitere Funktionalität zur Verfügung gestellt.
% Viele Pakete haben optionale Lade-Argumente, welche in eckigen Klammern angegeben werden.

% Nützliche Zusatzpakete sind zum Beispiel: graphicx (Grafiken einbinden), siunitx (Einheiten richtig formatieren), listings (Quellcode darstellen)

% Ein schönes Template gibt es hier: http://www.latextemplates.com/template/journal-article

\begin{document}
    \section{Grundlagen \LaTeX} 
    \subsection{Einführung}
        Diese Übersicht soll das Schreiben nachträglich erleichtern. Für weitere Details schaue bitte in die .tex Datei hinein!
        \LaTeX \@ ist eine Sammlung von Makros die in \textbf{TeX} programmiert sind. Das zu Grunde liegende \textbf{TeX}-System
        ist eine Programmiersprache. Es gibt entsprechend für den Anwender viele Elemente einer Programmiersprache um den Textsatz
        zu gestalten. Mithilfe von \textsl{Variablen} können alle Format-Eigenschaften eines Dokuments programmatisch geändert werden.
        Hierbei gibt es verschiedene \textsl{Variablentypen}. Wichtig sind die \textsl{Längen-} und \textsl{Zählervariablen}.
        Diese Variablen können über \LaTeX-Befehle verändert werden. Auf \textbf{TeX}-Ebene gibt es Kontrollstrukturen, also \textsl{if then}
        Anweisungen, \textsl{Schleifen} und noch vieles mehr.
    \subsection{Böse Sachen}
        Zuerst einmal Sachen die \textbf{NICHT} schön sind:
        \begin{enumerate}
            \item Manuelle Leerzeichen mit \textbackslash \textbackslash
            \item Andauernd \verb!\textbf{...}! verwenden. Lieber Betonungen (emph...)!
        \end{enumerate} 
    \subsection{Struktur}
        Der Textkörper wird mit Abschnitts-Befehlen strukturiert. Diese unterscheiden sich nach Typ des Dokumentes (z.B. \textbf{article}). Jedoch gelten diese Befehle für alle:
        \begin{itemize}
            \item \verb!\section!
            \item \verb!\subsection!
            \item \verb!\subsubsection!
        \end{itemize}
        Für \textbf{book} gilt wiederum:
        \begin{itemize}
            \item \verb!\part!
            \item \verb!\chapter!
            \item \verb!\paragraph!
            \item \verb!\subparagraph!
        \end{itemize}
    \subsection{Zeichen}
        Folgende Zeichen können \textbf{nicht} einfach so eingetippt werden:
        \begin{itemize}
            \item \& \@ \% \@ \$ \@ \# \@ \_ \@ \{ \@ \} \@ \~ \@ \^ \@
        \end{itemize}
        Diese können mit der Kombination aus dem \textbackslash \@ Befehl und dem jeweiligen Zeichen erzeugt werden (\em Escape-Befehl\normalfont).  % Hier wird der unäre Befehl genommen (Schalter)
        Um Leerzeichen zu forcieren kann \verb!\@! benutzt werden. Ein Bindestrich kann mit \verb!--! eingefügt werden. Mit dem Tilde Symbol kann ein gesperrtes Leerzeichen erzeugt werden (verhindert einen Zeilenumbruch).
    \subsection{Textformatierungen}
        \LaTeX \@ unterstützt folgende physische Textformatierungen:
        \begin{itemize}
            \item \verb!\textbf{bold face}! -- \textbf{bold face}
            \item \verb!\textit{italics}! -- \textit{italics}
            \item \verb!\textsl{slanted}! -- \textsl{slanted}
            \item \verb!\textsc{small caps}! -- \textsc{small caps}
        \end{itemize}
    \subsection{Schriftgrößen}
        Folgende lokalen Schriftgrößenänderungen sind in \LaTeX \@ möglich:
    \begin{enumerate}
        \item \verb!\tiny!
        \item \verb!\scriptsize!
        \item \verb!\footnotesize!
        \item \verb!\small!
        \item \verb!\normalsize!
        \item \verb!\large!
        \item \verb!\Large!
        \item \verb!\LARGE!
        \item \verb!\huge!
        \item \verb!\Huge!
    \end{enumerate}
    Um die Schriftgröße für das gesamte Dokument zu ändern gibt es Präambel-Befehle.
    \subsection{Umgebungen}
        Eine Umgebung schließt einen Text mit \verb!\begin{Umgebung}! und \verb!\end{Umgebung}! ein. Innerhalb der Umgebung gelten eigene und spezielle Regeln. Wichtige Beispiele sind:
    \begin{itemize}
        \item enumerate (Aufzählung)
        \item itemize (Liste)
        \item equation (mathematische Gleichung)
        \item description (Lexikon ähnliche Aufzählung)
    \end{itemize}
    \subsection{Listings}
        Listings enthalten nicht-formatierten Text, der wichtig zur Darstellung von Quellcode jeglicher Programmiersprachen ist.
    \lstset{language=C}
    \begin{lstlisting}[frame=single]
        int main(int argc, char* argv[]) 
        {
            printf("Hello World!");
        }
    \end{lstlisting}
    Mit den Optionen \verb!\lstset{language=C}!, sowie \verb!\begin{lstlisting}[frame=single]! können die Programmiersprache festgelegt und der Textrahmen aktiviert werden.
    Möchte man jedoch nur eine Textbox erstellen, so kann man dies auch mit dem \verb!\fbox! Befehl tun.

    \subsection{Externe Dokumente}
        Mithilfe des \verb!\input{file}! Befehls können weitere \emph{.tex} Dateien in das Dokument so eingefügt werden, als würde der Inhalt der Datei im Originaldokument stehen.
        Dabei können die Pfadangaben \emph{relativ} oder \emph{absolut} sein. Den Unterdokumenten muss eine extra Zeile am Anfang eingefügt werden, die auf das Master Dokument hinweist. Alternativ kann im \emph{TexMaker}
        das Master Dokument über das Menü Optionen eingestellt werden.
    \subsection{Längen}
        Mithilfe des \verb!\setlength{variable}{länge}! Befehls können die in \textbf{TeX} auftretenden Längen verändert werden. \textbf{Alle Längen haben eine Einheit!}
        Wichtige Variablen sind zum Beispiel:
    \begin{itemize}
        \item \verb!\parindent! - Definiert den Einzug eines Absatzes
        \item \verb!\parskip! - Definiert den Abstand zwischen zwei Absätzen
        \item \verb!\baselineskip! - Der Zeilenabstand
    \end{itemize}
    \textbf{Anwendungsbeispiele:}
    \begin{enumerate}
        \item \verb!\setlength{\parindent}{10pt}! - Einzug auf 10pt setzen
        \item \verb!\setlength{\parskip}{0.5\baselineskip}! - Der Abstand zwischen zwei Absätzen wird auf die Höhe einer halben Zeile gesetzt.
    \end{enumerate}
    \textsl{Relative Längenangaben} (siehe 2. Beispiel) beziehen sich auf einen Referenzwert. Dieser Bezug erfolgt mithilfe einer Multiplikation (in \LaTeX \@ wird dazu der Multiplikator einfach direkt an den Referenzwert geschrieben). Eine Addition geht nicht. Das wäre Stretching!
    \textsl{Absolute Längenangaben} werden mit Maßzahl und Einheit (\textbf{WICHTIG!}) direkt angegeben. Es ist sinnvoll \textbf{immer relative Längenangaben} zu nutzen, denn bei einer Änderung des Schriftgrads von 10pt auf bspw. 12pt ändert sich der Abstand zwischen den Absätzen genau richtig mit.
    Auch erweist sich dies bei der Breite von Bildern und bei der Spaltenweite von Tabellen als besonders hilfreich. Mithilfe des \verb!\the! Befehls können Längen (Wert der Variablen) angezeigt werden.
    Es gibt viele verschiedene Längeneinheiten:
    \begin{itemize}
        \item Punkt - \textbf{pt} - entspricht: \textbf{1pt}
        \item Millimeter - \textbf{mm} - entspricht: \textbf{2.84pt}
        \item Zentimeter - \textbf{cm} - entspricht: \textbf{28.4pt}
        \item Inch - \textbf{in} - entspricht: \textbf{72.27pt}
    \end{itemize}
    \textbf{Wichtiger Tipp:}
    \\
    Die wichtigsten Längen für die Gestaltung einer Seite können über den \verb!\layout{}! Befehl angezeigt werden. Dazu muss aber vorher das Package \textsl{layout} mit \verb!\usepackage{}! geladen werden.
    Interessante weitere Längen sind:
    \begin{itemize}
        \item \verb!\columnsep!, \verb!\columnwidth! - Bei der mehrspaltigen Gestaltung
        \item \verb!\tabcolsep! - Für Tabellen
        \item \verb!\itemsep! - Für Listen
    \end{itemize}
    Längen lassen außerdem über diverse Befehle verändern:
    \begin{itemize}
        \item \verb!\setlength{\länge}{wert}! - Setzt eine Länge auf einen Wert
        \item \verb!\addtolength{\länge}{wert}! - Addiert eine Länge zu einem Wert
        \item \verb!\settowidth{\länge}{beispieltext}! - Breite des Beispieltexts auf \verb!\länge! setzen
        \item \verb!\settoheigth{\länge}{beispieltext}! - Höhe des Beispieltexts auf \verb!\länge! setzen
        \item \verb!\settodepth{\länge}{beispieltext}! - Tiefe des Beispieltexts auf \verb!\länge! setzen
    \end{itemize}
    Bei letzeren Befehlen wird der Beispieltext jedoch \textbf{NICHT} angezeigt! Interessant ist, dass der Beispieltext selbst das Ergebnis eines Befehls sein kann.
    \\
    \\
    \\
    Um ein schöneres Layout zu erreichen streckt \LaTeX \@ einige Größen. Hierzu zählen der Abstand zwischen Wörtern für Blocksätze und die Abstände zwischen den Absätzen um die Seite optimal auszunutzen. Wenn das nicht klappt, meckert \LaTeX \@ über eine \textsl{overfull} oder \textsl{underfull} \verb!\vbox!.
    \textbf{Eigene Längen definieren:}
    Eigene Längen können mit dem Befehl \verb!\newlength! definiert werden. Dies kann zum Beispiel bei den Spaltenbreiten von Tabellen im ganzen Dokument, oder für eine einheitliche Breite von Bildern (automatisch skaliert) nützlich sein.
    Die Verwendung geschieht dabei wie folgt:
    \lstset{language=TeX}
    \begin{lstlisting}[frame=single]
        \newlength{\imgWidth}
        \setlength{\imgWidth}{0.8\textwidth}
    \end{lstlisting}
    Zuerst wird die Länge \textsl{imgWidth} mit \verb!\newlength! erzeugt und dann mit den Längenänderungsbefehlen gesetzt.
    Alle Bilder im Text können mit dieser Variablen auf 80\% der Textbreite skaliert werden. Um alle Bilder neu zu skalieren, muss nur die 0.8 geändert werden.

    \subsection{Befehle}
        \subsubsection{Eigene Kommandos}
            Mit dem Befehl \verb!\newcommand! lassen sich eigene Befehle definieren.
            % Interessant: Gesperrtes Leerzeichen, um Umbruch zu verhindern (2. Befehl in lstlisting).
            \lstset{language=TeX}
            \begin{lstlisting}[frame=single]
                \newcommand{\name}{definition}
                \newcommand{\zB}{z.\,B.\,\@}
                \newcommand{\abk}{Abkuerzung}
            \end{lstlisting}
            In der hier gezeigten Form dienen sie als Abkürzung für häufig auftretende Befehle, die sonst umständlich einzutippen wären.
            Die fortgeschrittene Variante des \verb!\newcommand! Befehls gebraucht \textbf{Argumente}. So wird der selbst-definierte Befehl zur Funktion.
            \lstset{language=TeX}
            \begin{lstlisting}[frame=single]
                \newcommand{\macheFett}[1]{\textbf{#1}}
            \end{lstlisting}
            In der Definition wird über \#1 auf das erste Argument zugegriffen, über \#2 auf das zweite Argument, etc. Dies ist für \textbf{maximal 9 Argumente} möglich!
            Möchte man bestehende Befehle verändern, so kann man dies mit dem \verb!\renewcommand! Befehl tun. Die Syntax ist dabei die gleiche, wie bei \verb!\newcommand!. Hierbei können auch \textbf{Argumente} übergeben werden!
            \lstset{language=TeX}
            \begin{lstlisting}[frame=single]
                \renewcommand{\name}[1]{definition}
                \renewcommand{\em}{\bfseries}
            \end{lstlisting}
            Bei dem Beispiel mit dem \verb!\em! geht allerdings durch die Erneuerung die Schalter-Logik verloren.
        \subsubsection{Zähler}
            Zähler sind wie \textbf{Längen} ein Variablen-Typ mit denen in \LaTeX \@ programmiert werden kann. Mit Zählern regelt \LaTeX \@ (fast) alle Verweise im Dokument. Wie bei \textbf{Längen} können eigene Zähler definiert werden.
            Diese können auch mit diversen Befehlen bearbeitet werden. Hierbei gibt es eine Liste vordefinierter Zähler:
            \begin{itemize}
                \item part
                \item chapter
                \item section
                \item subsection
                \item subsubsection
                \item paragraph
                \item subparagraph
                \item page
                \item equation
                \item figure
                \item table
                \item footnote
                \item mpfootnote
            \end{itemize}
            Für Listen gelten außerdem diese Zähler:
            \begin{itemize}
                \item enumi
                \item enumii
                \item enumiii
                \item enumiv
            \end{itemize}
            Beispiel:
            \lstset{language=TeX}
            \begin{lstlisting}[frame=single]
                \stepcounter{enumi}
                \addtocounter{section}{zahl}
                \setcounter{equation}{zahl}

                \newcounter{numDoener}

                \thenumDoener % Formatierter Text

                \value{numDoener} % Wert unformatiert
            \end{lstlisting}
            Der Befehl \verb!\the! zeigt die Zähler an (ohne \verb!\!).  Das Besondere an \verb!\value! ist, ist dass der unformatierte Wert zum Gebrauch in Rechnungen genutzt werden kann.
            \textbf{Zählerdarstellung:}
            Mithilfe folgender Befehle kann die Darstellung von Zahlen mithilfe des \verb!\renewcommand! Befehls verändert werden.
            \begin{enumerate}
                \item \verb!\arabic! -- 1. 2. 3. ...
                \item \verb!\roman! -- i. ii. iii. ...
                \item \verb!\Roman! -- I. II. III. ...
                \item \verb!\alph! -- a. b. c. ...
                \item \verb!\Alph! -- A. B. C. ...
            \end{enumerate}
            Zur Änderung \textbf{aller} Aufzählungen des Dokuments wird \verb!\theenumi! usw. in der Präambel neu definiert.
            Dies ist notwendig, da \LaTeX \@ anscheinend den Befehl indirekt zur Darstellung nutzt.
            \lstset{language = TeX}
            \begin{lstlisting}[frame = single]
                \documentclass{article}

                \renewcommand{\theenumi}{\Roman{enumi}}

                \begin{document}
                    \begin{enumerate}
                        ...
                    \end{enumerate}
                \end{document}
            \end{lstlisting}
            Zur Änderung einer einzigen Aufzählung wird \verb!\theenumi! usw. direkt in der entsprechenden Aufzählung neu definiert.
            \lstset{language = TeX}
            \begin{lstlisting}[frame = single]
                \documentclass{article}

                \begin{document}
                    \begin{enumerate}

                        \renewcommand{...}

                        ...
                    \end{enumerate}
                \end{document} 
            \end{lstlisting}
    \subsection{Querverweise}
        Querverweise verweisen auf einen anderen Textteil. Diese Querverweise sind automatisiert und gebrauchen die Zähler.
        Dazu wird mit \verb!\label! ein referenzierbarer Name erzeugt, der dann mit \verb!\ref! gebraucht werden kann.
        Hierbei hat es sich im allgemeinen durchgesetzt, vor dem eigentlichen Labelnamen eine Bezeichnung einzufügen,
        die diese Verweisart beschreibt. Für Verweise auf Abschnitte schreibt man dazu einfach ein '\textbf{sec:}' vor den Labelnamen.
        \lstset{language = TeX}
        \begin{lstlisting}[frame = single]
            \section{Stand der Technik} 
            \label{sec:standTechnik}
            .
            .
            .
            Wie in Abschnitt ~\ref{sec:standTechnik}...
        \end{lstlisting}
        Das Label muss hierbei \textbf{direkt hinter dem referenzierten Objekt stehen!}
        Weitere Labelgruppen lauten wie folgt:
        \begin{itemize}
            \item sec:Abschnitt
            \item fig:Abbildung
            \item table:Tabelle
            \item eqn:Gleichung
            \item fn:Fußnote
            \item item:Aufzählungspunkt
        \end{itemize}
        Wenn an Stelle des Zählerwertes die Seite referenziert werden soll, auf der das Objekt steht, so gebraucht man den \verb!\pageref! Befehl.
    \subsection{Gleitobjekte}
\end{document}